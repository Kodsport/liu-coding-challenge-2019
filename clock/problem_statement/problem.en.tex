\problemname{Clock}

Engineers at Spectre Imaging have decided to start investing in smart clock market. They are creating a prototype clock that will be able to show some images.

The clock's rectangular display consists of black-and-white pixels. A pixel can have two states: ON or OFF. Since it is a complicated process to construct such a display, Spectre Imaging came up with the idea to divide the pixels in groups, where all pixels in a group would either be ON or OFF.

Given the screen resolution of the clock and the set of pictures that must be shown on the display, what is the minimal number of groups that are needed do be able to display all of the pictures?

\section*{Input}

The first line contains integers $n$, $h$ and $w$ $(1 \leq n,h,w \leq 100)$ -- the number of images that must be shown, height of the display in pixel, and the width of the display in pixels respectively.

After that follows $kh$ lines with $w$ characters each. All of these lines contain either a '\texttt{*}' or a '\texttt{.}' which correspond to a pixel being ON or OFF respectively.

\section*{Output}

Output consists of a single integer: the minimal number of pixel groups into which the display can be divided into.

Explanation for sample input 1 (each of the 4 pixel groups): \newline
\texttt{1.. .2. ... ..4} \newline
\texttt{1.. ... .3. ..4} \newline
